\documentclass[a4paper, 12pt]{article} % report, book, letter
\usepackage{amsmath}
\usepackage{multicol}
\usepackage{lipsum}
\usepackage{textcomp}
\usepackage{lmodern}
\usepackage{multirow}
\usepackage{graphicx}
\usepackage{float}

\graphicspath{{images/}}
\renewcommand{\contentsname}{Indice}

\title{Esempio di Documento}
\author{Mario Rossi}
\date{\today}

\begin{document}
\maketitle
\tableofcontents % \renewcommand{\contentsname}{Indice}
\newpage

\section{Esempio di sezione}
Questa è una frase. \\
Questa è un'altra frase.

Questa è un'altra frase, in un altro paragrafo.
% \indent
% \noindent 

\section{Altro esempio di sezione}
\lipsum[1]
\subsection{Esempio di sottosezione}
\lipsum[2]
\subsubsection{Esempio di sotto-sottosezione}
%\ usepackage{multicol}
\begin{multicols}{3}
[
\begin{large}
Testo che verrà visualizzato per intero senza separazione in colonne.
\end{large} % \tiny, \small, \large, \huge
]
\lipsum[3-5]
\end{multicols}

% altre possibilità \chapter{} per book e report

\section{Altre opzioni}
\label{opzioni}
% caratteri speciali
30\textcelsius \\
100\texteuro % \usepackage{lmodern}

\noindent
Come detto in \ref{opzioni} \\
Come detto a pagina \pageref{opzioni} \\
Questa è una nota a piè di pagina\footnote{Esempio di nota} \\
Ciao, \textbf{Ciao}, \textit{Ciao}, \underline{Ciao} \\

\noindent
\begin{itemize}
    \item primo punto
    \item[-] secondo punto
\end{itemize}

\vspace{2cm}

\begin{enumerate}
    \item primo sottoelenco:
    \begin{itemize}
        \item primo punto
        \item secondo punto
    \end{itemize}
    \item secondo sottoelenco:
    \begin{description}
        \item[prova] terzo punto
        \item[ciao] quarto punto
    \end{description}
\end{enumerate}

\begin{flushleft}
\lipsum[1]    
\end{flushleft}

\begin{flushright}
\lipsum[1]    
\end{flushright}

\begin{center}
\lipsum[1]
\end{center}

\begin{center}
\begin{tabular}{|c|l|rrr}
\hline
nome & cognome & ciao & ciao & ciao \\
\hline
Mario & Rossi & a & b & c \\
Luca & Bianchi & c & d & e \\
\hline
\end{tabular}
\end{center}

% più complicata
\begin{center} % \usepackage{multirow}
{\renewcommand{\arraystretch}{1.5}\renewcommand{\tabcolsep}{0.5cm}
\begin{tabular}{|c|c|c|c|c|}
\hline
nome & cognome & \multicolumn{3}{|c|}{ciao} \\
\hline
\multirow{2}{*}{Mario} & Rossi & a & b & c \\ \cline{3-5}
& Bianchi & c & d & e \\
\hline
\end{tabular}
}
\end{center}

% per tabelle in excel o csv o txt usare LaTeX table convert.

% \usepackage{graphicx}
\begin{figure}[!htb] % ! per non considerare parametri interni che ostruiscono
    \centering
    \includegraphics[width=10cm]{ai-sf.png}
    \caption{Logo di AISF}
    \label{fig:aisf}
\end{figure}

% \renewcommand{\figurename}{Figura} % per sostituire il Figure (va messo prima)

% height = 5cm
% width = 0.5\textwidth
% width = 0.5\linewidth
% se si è disperati \usepackage{float} [H] 

\noindent
Come visto in figura \ref{fig:aisf}.

% più immagini accostate

\begin{figure}[H]
\begin{minipage}{.5\textwidth}
\centering
\includegraphics[width=6cm]{ai-sf.png}
\caption{primo logo}
\label{fig:logo1}
\end{minipage}
\begin{minipage}{.5\textwidth}
\centering
\includegraphics[width=6cm]{ai-sf.png}
\caption{secondo logo}
\label{fig:logo2}
\end{minipage}
\end{figure}

\begin{figure}[H]
\hspace{-2.25cm}
\begin{minipage}{.5\textwidth}
\centering
\includegraphics[width=9cm]{ai-sf.png}
\end{minipage}
\hspace{2cm}
\begin{minipage}{.5\textwidth}
\centering
\includegraphics[width=9cm]{ai-sf.png}
\end{minipage}
\end{figure}

\begin{figure}[H]
\begin{minipage}{.4\textwidth}
\centering
\includegraphics[width=5cm]{ai-sf.png}
\caption{primo logo}
\label{fig:logo1}
\end{minipage}
\begin{minipage}{.6\textwidth}
\centering
{\renewcommand{\arraystretch}{1.5}\renewcommand{\tabcolsep}{0.5cm}
\begin{tabular}{|c|c|c|c|c|}
\hline
nome & cognome & \multicolumn{3}{|c|}{ciao} \\
\hline
\multirow{2}{*}{Mario} & Rossi & a & b & c \\ \cline{3-5}
& Bianchi & c & d & e \\
\hline
\end{tabular}
}
\caption{prima tabella} % attenzione a nomenclatura!
\label{tab:tabella}
\end{minipage}
\end{figure}

% inserire ora ai-sf.png in folder images
% modificare in images/ai-sf.png
% a inizio documento \graphicspath{{images/}}

\end{document}
